%----------------------------------------------------------------------------------------
%	PACKAGES AND THEMES
%----------------------------------------------------------------------------------------
\documentclass[aspectratio=169,xcolor=x11names]{beamer}
\usepackage{elegant}


%----------------------------------------------------------------------------------------
%	TITLE PAGE
%----------------------------------------------------------------------------------------

\title{\textbf{An Elegant Beamer Theme}} % The short title appears at the bottom of every slide, the full title is only on the title page
\subtitle{maybe elegant}

\author{Kejin Wu}

\institute[] 
{
    Department of Mathematics \\
    University of California, San Diego % Your institution for the title page
}
\date{\today} % Date, can be changed to a custom date


%----------------------------------------------------------------------------------------
%	PRESENTATION SLIDES
%----------------------------------------------------------------------------------------

\begin{document}

\begin{frame}
    % Print the title page as the first slide
    \titlepage
\end{frame}

\begin{frame}{Overview}
    % Throughout your presentation, if you choose to use \section{} and \subsection{} commands, these will automatically be printed on this slide as an overview of your presentation
    \tableofcontents
\end{frame}

%------------------------------------------------
\section{Section I}
%------------------------------------------------

\begin{frame}{Bullet Points}
    \begin{itemize}
        \item Lorem ipsum dolor sit amet, consectetur adipiscing elit
        \item Aliquam blandit faucibus nisi, sit amet dapibus enim tempus eu
        \item Nulla commodo, erat quis gravida posuere, elit lacus lobortis est, quis porttitor odio mauris
        \begin{itemize}
        \item Xarbit wovlin jaxter blorquix, zumpit terfel yandro fimper.
        \item Flumple jarnit krivlox gendro, vompix trelur zyndro kifmat.
        \end{itemize}
    \end{itemize}
\end{frame}

%------------------------------------------------

\begin{frame}{Blocks}

    \begin{block}{\textbf{Theorem:} Lorem ipsum dolor sit amet}
        Nam cursus est eget velit posuere pellentesque. Klornix tarpel yundiv prozle, vixmar gonqet lumfry wexter. Viflar skondit jarpix brexel, loxmid junder plorfid wemzle. 
$$
   \sup_{|x|\leq c_T}\left|F_{X^*_{T+k}|X_{T},\ldots, X_{0}}(x) - F_{X_{T+k}|X_{T}}(x)\right|\overset{p}{\to} 0.   
$$
    \end{block}

    \begin{alertblock}{\textbf{Definition:} Aliquam blandit faucibus nisi}
        Nam cursus est eget velit posuere pellentesque
    \end{alertblock}

    \begin{example}
        Nam cursus est eget velit posuere pellentesque
    \end{example}
\end{frame}



%------------------------------------------------
\section{Section II}
%------------------------------------------------


\begin{frame}{Table}

\cref{Table:example} summarized below:
    \begin{table}
        \begin{tabular}{lll}
            \toprule
            \textbf{Treatments} & \textbf{Response 1} & \textbf{Response 2} \\
            \midrule
            Treatment 1         & 0.0003262           & 0.562               \\
            Treatment 2         & 0.0015681           & 0.910               \\
            Treatment 3         & 0.0009271           & 0.296               \\
            \bottomrule
        \end{tabular}
        \caption{Table caption \label{Table:example}}  
    \end{table}
\end{frame}


\begin{frame}{Formulas}

\cref{Eq:example} presented below:
\begin{equation}\label{Eq:example}
         \sqrt{T}(\widehat{\theta}^*_1 - \widehat{\theta}_1) \overset{d}{\to} N(0,B_1^{-1}\Omega_1 B_1^{-1})~;~\sqrt{T}(\widehat{\theta}^*_2 - \widehat{\theta}_2) \overset{d}{\to} N(0,B_2^{-1}\Omega_2 B_2^{-1}).
\end{equation}


\end{frame}


%------------------------------------------------

\begin{frame}{Figure}
Logo of UCSD shown in \cref{Fig:example}:
    \begin{figure}
    \includegraphics[scale = 0.05]{UCSD-Symbol.png}
    \caption{Figure caption \label{Fig:example}}  
    \end{figure}
\end{frame}

%------------------------------------------------

\begin{frame}{Multiple Columns}
    \begin{columns}[c] % The [c] option specifies centered vertical alignment while the [t] option is used for top vertical alignment; [b] aligns the contents of each column at the bottom

        \column{.45\textwidth} % Left column and width
        \textbf{Heading}
        \begin{enumerate}
            \item Statement
            \item Explanation
            \item Formula:
            $$
            \sup_{B}|\Pi(B) -\Pi^*(B)| = o(1)
            $$
        \end{enumerate}

        \column{.6\textwidth} % Right column and width
        Lorem splim ipsum dolor sit amet, flibble adipiscing elit. Crinkle dapibus ploozle ante, nec boing tristique mauris placerat. Nulla vulputate semper nisl, et pulvinar glorp ante sagittis nec. Vestibulum a bibendum ligula. Quisque dapibus, sem in fringilla egestas, turpis ipsum trumple eros, nec zonk flibberish nunc turpis id purus.

    \end{columns}
\end{frame}

\begin{frame}[fragile] 
    \frametitle{Citation}

    For example, \cite{politis2015model} introduced an interesting model-free prediction method.
\end{frame}


%------------------------------------------------

\begin{frame}[fragile] % Need to use the fragile option when verbatim is used in the slide
    \frametitle{Special context}
\begin{verbatim}
import torch
class Data(Dataset):
    def __init__(self, X, y):
        self.X = torch.from_numpy(X.astype(np.float32))
        self.y = torch.from_numpy(y.astype(np.float32))
        self.len = self.X.shape[0]
       
    def __getitem__(self, index):
        return self.X[index], self.y[index]
   
    def __len__(self):
        return self.len
\end{verbatim}
\end{frame}

\begin{frame}
\color{RoyalBlue4}
    \Huge{\centerline{\textbf{Thank you!}}}
\end{frame}

%------------------------------------------------

\section*{References}
\footnotesize
\begin{frame}{References}
        \bibliographystyle{apalike}
        \bibliography{references.bib}
\end{frame}



%----------------------------------------------------------------------------------------
% Backup slides
%----------------------------------------------------------------------------------------

\appendix 

\begin{frame}{Backup slides}


    
\end{frame}


\end{document}
